\documentclass[portuges,a4paper]{article}
\usepackage{babel}
\usepackage[mathletters]{ucs}
\usepackage[utf8x]{inputenc}

\usepackage[T1]{fontenc}
\usepackage{fancyvrb}
\usepackage{url}
\def\caixa#1{ \vskip .1cm \fbox{\begin{minipage}{0.95\columnwidth}#1\end{minipage}}\vskip .1cm}
\parindent 0pt
\parskip 6pt
\begin{document}

\title{Python}
\author{J.João and spln2018}
\date{\today}
\maketitle
\tableofcontents

\section{Python}

Recomenda-se:
\begin{itemize}
\item  \url{https://www.python-course.eu/python3_course.php}
\item \url{https://ehmatthes.github.io/pcc/cheatsheets/README.html}
\end{itemize}

\subsection{Números}

\caixa{
\textbf{Descrição}  \hfill{}  \textbf{Notação} \ \  \\
\texttt{int(exp)}\\
\texttt{12 0x12 0o12 0b1011101} \\
\texttt{hex(x) oct(x) bin(x) : integer → string} \\
\texttt{3+4j} \dotfill{} números complexos\\
\texttt{13//24} \dotfill{} divisão inteira (floor div)
}

\subsection{Strings}

\caixa{
\textbf{Descrição}  \hfill{}  \textbf{Notação} \ \  \\
\texttt{str(exp)} \\
\texttt{s1 + s2} \\
\texttt{s * 4}  \dotfill{} 4 repetições de s\\
\texttt{s[0]}  \dotfill{} first char\\
\texttt{s[-1]}  \dotfill{} last char\\
\texttt{s[3:5]}  \dotfill{} string slicing\\
\texttt{""" .... """ } \dotfill{} multiline string\\
\textbf{Chars}\\
\texttt{\textbackslash{}xDD} \dotfill{} \\
\texttt{\textbackslash{}uxxxx} \dotfill{} \\
\texttt{\textbackslash{}Uxxxxxxxx} \dotfill{} \\
\texttt{\textbackslash{}N\{name\}} \dotfill{} unicode's char name \\
\textbf{Métodos}\\
\texttt{s.split(sep)} \dotfill{} str × str → [str]\\
\texttt{s.join(str-seq)} \dotfill{} \\
}

\subsection{Funções  and Lambda}

\subsection{Tuplos}

\begin{Verbatim}
ponto=(12,12)
print(ponto[0])
\end{Verbatim}

\caixa{
\textbf{Descrição}  \hfill{}  \textbf{Notação} \ \  \\
\texttt{tuple(exp)} \\
}

\subsection{Listas}
\caixa{
\textbf{Descrição}  \hfill{}  \textbf{Notação} \ \  \\
list(exp)\\
reversed(list)\\
sorted(list)\\
max(list|dict|tuplos)\\
\texttt{zip(l1,l2)} \dotfill{}  lista de tuplos\\
list[3:5] \dotfill{} sub-lista\\
list[-3:] \dotfill{} últimos 3 \\
\texttt{[f(a) for a in range(20)]} \dotfill lista em compreensão\\
\textbf{métodos}\\
\texttt{l.append(l)}  \dotfill{}   \\
\texttt{l.remove(val)}  \dotfill{}   \\
\texttt{del l[i]}  \dotfill{}  delete by index \\
\texttt{l.insert(1,val)}  \dotfill{} delete by value  \\
\texttt{l.pop()}  \dotfill{}   \\
\texttt{l.pop(0)}  \dotfill{}  ... shift \\
\texttt{len(l)}  \dotfill{}   \\
\texttt{l.sort()}  \dotfill{}  ordena a lista  \\
\texttt{l.sort(reverse=True)}  \dotfill{}   \\
\texttt{l.reverse()}  \dotfill{}   \\
\texttt{min,max,sum}  \dotfill{}   \\
\texttt{}  \dotfill{}   \\
\texttt{}  \dotfill{}   \\
\texttt{}  \dotfill{}   \\
}

\subsection{Dicionários}

Chaves: atómicas ou tuplos

\begin{Verbatim}
en_pt = {"red" : "vermelho", "green" : "verde", "blue" : "azul"}
en-pt["yellow"]="amarelo"
print( en_pt )
print( en_pt["red"])
\end{Verbatim}

\caixa{
\textbf{Descrição}  \hfill{}  \textbf{Notação} \ \  \\
\texttt{dict(exp)} \\
\texttt{d=\{12:23, 23:33\}} \dotfill{} dic. em extensão\\
\texttt{d[12]=13} \dotfill{} set a value\\
\texttt{len(d)} \dotfill{} returns the number of stored pairs\\
\texttt{del d[k]} \dotfill{}  deletes the key k together with his value\\
\texttt{k in d} \dotfill{}  True, if a key k exists in the dictionary d\\
\texttt{k not in d} \dotfill{}  True, if a key k not in in the dictionary d\\
\texttt{\{a:f(a) for a in range(20)\}} \dotfill dic. em compreensão\\
\textbf{métodos e funções}\\
\texttt{d=dict(zip(l1,l2))} \dotfill{} dict from 2 lists\\
\texttt{is=en\_pt.items()} \dotfill{} lista de tuplos)\\
\texttt{d=en\_pt.copy()} \dotfill{} dict \\
\texttt{ks=en\_pt.keys()} \dotfill{} lista das chaves\\
\texttt{en\_pt.update(d)} \dotfill{} en\_pt plus d\\
\texttt{vs=en\_pt.values() \dotfill{}  lista das valores}\\
}

\subsection{Files}

\subsection{Classes, Objectos}

\subsection{Pacotes}

\subsection{alguns design patterns}

\subsection{Expressões regulares}

\subsection{Parsers}

\subsection{Processos}

\subsection{Modulos importantes}

\subsection{ nltk}
\end{document}

